\documentclass[aspectratio=169]{beamer}

\usepackage{centernot}
\beamertemplatenavigationsymbolsempty
\title{Das Trivialkriterium}
\date{}

\newcommand*{\N}{\mathbb N}

\begin{document}
  \begin{frame}
    \titlepage
  \end{frame}

  \begin{frame}
    \begin{theorem}[Trivialkriterium]
      Wenn die Reihe $\sum_{k=1}^\infty a_k$ konvergiert, dann ist die Reihe $(a_k)_{k\in\N}$ eine Nullfolge.
    \end{theorem}
  \end{frame}

  \begin{frame}
    $\sum_{k=1}^\infty a_k \text{ konvergiert} \implies (a_k)_{k\in\N} \text{ ist eine Nullfolge.}$
  \end{frame}

  \begin{frame}
    \frametitle{Beispiele}

    Die Reihe $\sum_{k=1}^\infty (-1)^k$ divergiert nach dem Trivialkriterium.
  \end{frame}

  \begin{frame}
    \frametitle{Beispiele}

    Die Reihe $\sum_{k=1}^\infty \sqrt[k]{4}$ divergiert nach dem Trivialkriterium.
  \end{frame}

  \begin{frame}
    \frametitle{Warnung}

    $\lim_{k\to\infty} a_k = 0 \centernot\implies \sum_{k=1}^\infty a_k \text{ konvergiert}$
  \end{frame}
\end{document}
